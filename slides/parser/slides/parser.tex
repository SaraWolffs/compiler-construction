\documentclass
  [ department=icis
  , notes={show notes}
  , slidesperpage=1
  , nohandout
  , official=true
  ] {beamerruhuisstijl}

% Use Times Roman, with corresponding math font
\usepackage{mathptmx}
\usefonttheme{serif}
%Add Idris bindings
\usepackage{fancyvrb}
%\usepackage[usenames]{xcolor}
\newcommand{\IdrisData}[1]{\textbf{\textcolor{red}{#1}}}
\newcommand{\IdrisType}[1]{\textbf{\textcolor{blue}{#1}}}
\newcommand{\IdrisBound}[1]{\textbf{\textcolor{purple}{#1}}}
\newcommand{\IdrisFunction}[1]{\textbf{\textcolor[RGB]{30,150,10}{#1}}}
\newcommand{\IdrisKeyword}[1]{{\underline{#1}}}
\newcommand{\IdrisImplicit}[1]{{\itshape \IdrisBound{#1}}}

\newcommand{\frameAndSection}[1]{\section{#1}\begin{frame}{#1}}
\newcommand{\fframeAndSection}[1]{\section{#1}\begin{frame}[fragile]{#1}}

\usepackage[english]{babel}

%\usepackage{tikz}
%\usetikzlibrary{positioning}

% needs debian package texlive-math-extra
%\usepackage{stmaryrd} % for \llbracket, \rrbracket (scott brackets)

% to get rid of the error "No room for a new \dimen" caused by some packages
\usepackage{etex}
\usepackage[retainorgcmds]{IEEEtrantools}
%\usepackage{bussproofs}

\usepackage{listings}
\usepackage{courier}
\lstset{language=C,basicstyle=\ttfamily}

%% Remove the navigation symbols
\setbeamertemplate{navigation symbols}{}


%\def\redb{\rightarrow_{\beta}}
%\def\yields{\vdash}
%\def\oftype{{:}}
%\def\reddb{\twoheadrightarrow_{\beta}}
%\newcommand{\SL}{\textsf{SL}}
%\def\FV#1{\textsf{FV}(#1)}
%\def\TV#1{\textsf{TV}(#1)}
%\def\TFA#1#2{\forall{#1}.{#2}}

%\def\ite#1#2#3{\kw{if}~{#1}~\kw{then}~{#2}~\kw{else}~{#3}}
%\def\letexpr#1#2#3{\kw{let}~{#1}={#2}~\kw{in}~{#3}}

%\renewcommand{\leq}{\leqslant}
%\newcommand{\kw}[1]{\textbf{\small{#1}}}



% Adjust formatting of itemize lists according to some recommendations I once
% found somewhere on the internet.
% - no bullets, no indentation
% - more spacing
% See section 12.1 of the beamer user guide.

% removes the bullets of itemize items of all depths
\setbeamertemplate{itemize items}{}

% spacing between items
\newcommand{\myitemsep}{4pt}%

% beamerruhuisstijl redefines itemize environments (and fucks it up IMHO)
% This redefines itemize and adjusts spacing
\makeatletter
\renewcommand{\itemize}[1][]{%
  \beamer@ifempty{#1}{}{\def\beamer@defaultospec{#1}}%
  \ifnum \@itemdepth >2\relax\@toodeep\else
    \advance\@itemdepth\@ne
    \beamer@computepref\@itemdepth% sets \beameritemnestingprefix
    \usebeamerfont{itemize/enumerate \beameritemnestingprefix body}%
    \usebeamercolor[fg]{itemize/enumerate \beameritemnestingprefix body}%
    \usebeamertemplate{itemize/enumerate \beameritemnestingprefix body begin}%
    % removes indentation of top-level lists
    \setlength{\leftmargini}{0pt}% this has to go here for some reason
    \list
      {\usebeamertemplate{itemize \beameritemnestingprefix item}}
      {\def\makelabel##1{%
          {%
            \hss\llap{{%
                \usebeamerfont*{itemize \beameritemnestingprefix item}%
                \usebeamercolor[fg]{itemize \beameritemnestingprefix item}##1}}%
          }%
        }%
        % increases spacing between list items
        % these have to go here for some reason
        \setlength{\itemsep}{\myitemsep}%
        \setlength{\partopsep}{-\myitemsep}%
        \setlength{\topsep}{\myitemsep}%
      }
  \fi%
  \beamer@cramped%
  \raggedright%
  \beamer@firstlineitemizeunskip%
}
\makeatother

\begin{document}


\title{Parsing in Idris}
\subtitle{A dependently typed functional language}
\author{Sal~Wolffs}
\date{\today}

\begin{frame}
  % suppress overfull hbox error message for the title page
  \hfuzz=\maxdimen
  \titlepage
  \maketitle
\end{frame}

\frameAndSection{Parsing in Idris}
  A dependently typed functional language
\end{frame}


\frameAndSection{Idris}
\onslide<+->
\begin{itemize}
    \onslide<+->\item Focus: general-purpose programming.
    \onslide<+->\item Haskell-like.
    \onslide<+->\item Type system: $\lambda\Pi 2$ w/ full polymorphism.
    \onslide<+->\item Additionally has ad-hoc polymorphism/overloading.
    \onslide<+->\item Optional totality checking.
    \onslide<+->\item Evaluation order: default strict.
    \onslide<+->\item Undecidable: totality, type inference.
    \onslide<+->\item Standard libraries under active development.
\end{itemize}
\end{frame}

\frameAndSection{Technique}
\onslide<+->
\begin{itemize} 
    \onslide<+->\item Two-phase: lex, then use parser combinators.
    \onslide<+->\item Idris has Text.Lexer and Text.Parser.
    \onslide<+->\item Both use a deep grammar DSL.
    \onslide<+->\item Lexer grammar is context-free (Applicative).
    \onslide<+->\item Parser grammar is context-sensitive (Monad).
    \onslide<+->\item Lexer sequentially tries \textbf{consuming} token grammars.
    \onslide<+->\item Parser is a parser combinator lib like Parsec.
\end{itemize}
\end{frame}

\frameAndSection{Progress}
\onslide<+->
\begin{itemize}
    \onslide<+->\item Lexer works
    \onslide<+->\item AST is defined
    \onslide<+->\item Parsing is WIP
\end{itemize}
\end{frame}

\frameAndSection{Grammar changes: front}
\onslide<+->
\begin{itemize}
    \onslide<+->\item Generally slightly more C-like.
    \onslide<+->\item Compound statements are just statements.
    \onslide<+->\item Control structures affect any statement.
    \onslide<+->\item Any expression can be a statement.
    \onslide<+->\item The empty function is legal (:: -> Void).
    \onslide<+->\item String is a type, and literals exist.
\end{itemize}
\end{frame}

\frameAndSection{Grammar changes: back}
\onslide<+->
\begin{itemize}
    \onslide<+->\item Some refactoring of left-recursion.
    \onslide<+->\item Int literals are positive.
    \onslide<+->\item Operators and expressions have \textbf{levels}.
    \onslide<+->\item Lower levels bind more strongly. 
    \onslide<+->\item Associativity is enforced by level bias. 
\end{itemize}
\end{frame}

\frameAndSection{AST}
\onslide<+->
\begin{itemize}
    \onslide<+->\item Multi-type AST, typecheck guarantees legal parse.
    \onslide<+->\item Every non-terminal is a datatype (as are some terminals).
    \onslide<+->\item Every production rule is a constructor.
    \onslide<+->\item The grammar has been inlined slightly in translation.
    \onslide<+->\item We want annotations! Now, location, later, types.
    \onslide<+->\item Every constructor has a final "note" argument.
    \onslide<+->\item Reuse: polymorphic in noteType.
\end{itemize}
\end{frame}

\fframeAndSection{AST code example: Expression levels}
\onslide<+->
\onslide<+->
Let's take a look at a stripped-down version of the AST for Expr.
\onslide<+->
% START CODE
\begin{Verbatim}[commandchars=\\\{\}]
\IdrisKeyword{data} \IdrisType{Expr} : (\IdrisType{\IdrisBound{n}:Nat}) -> \IdrisType{Type} \IdrisKeyword{where}
  \IdrisData{LOpExpr} : \IdrisType{Expr} (\IdrisData{S} \IdrisImplicit{n})->\IdrisType{LOp} \IdrisImplicit{n}->\IdrisType{Expr} \IdrisImplicit{n}->\IdrisType{Expr} (\IdrisData{S} \IdrisImplicit{n}\IdrisFunction{+}\IdrisImplicit{k})
  \IdrisData{LitNum} : \{\IdrisImplicit{k}:\IdrisType{Nat}\} -> \IdrisType{Nat} -> \IdrisType{Expr} (\IdrisData{0}\IdrisFunction{+}\IdrisImplicit{k})
  \IdrisData{ParenExpr} : \{\IdrisImplicit{k}:\IdrisType{Nat}\} -> \IdrisType{Expr} \IdrisImplicit{n} -> \IdrisType{Expr} (\IdrisData{0}\IdrisFunction{+}\IdrisImplicit{k})
  \IdrisData{UnOpExpr} : \{\IdrisImplicit{k}:\IdrisType{Nat}\} -> \IdrisType{UnOp} \IdrisImplicit{n} -> \IdrisType{Expr} \IdrisImplicit{n} -> \IdrisType{Expr} (\IdrisImplicit{n}\IdrisFunction{+}\IdrisImplicit{k})
\end{Verbatim}
% END CODE
\onslide<+->
\begin{itemize}
    \item \IdrisImplicit{k} is a "slack variable", turning strict equalities into bounds.
\onslide<+->
    \item Note: \{\IdrisImplicit{k}:\IdrisType{Nat}\} can be (and in reality is) left out.
\onslide<+->
    \item Importantly, \IdrisType{LOp} constructors produce fixed levels.
\onslide<+->
    \item \IdrisData{Plus}:\IdrisType{LOp} \IdrisData{6} and
        \IdrisData{Mult}:\IdrisType{LOp} \IdrisData{4} guarantee
        \IdrisData{Mult} binds more strongly.
\onslide<+->
    \item Reality: more constructors, annotation, proper defined constants.
\onslide<+->
    \item Additionally, \IdrisData{LitNum} is part of \IdrisData{Lit}.
\end{itemize}
\end{frame}

\frameAndSection{Parsing left-recursion}
\onslide<+->
\begin{itemize}
    \onslide<+->\item Left recursion context-free is hard...
    \onslide<+->\item Parser combinators are context-\textbf{sensitive}.
    \onslide<+->\item Parsing weaker grammars using stronger ones is easy.
    \onslide<+->\item Parse left side first, bind as context parsing rest.
    \onslide<+->\item \IdrisFunction{chainl} combinator does this, using
        \IdrisFunction{>>=} . 
    \onslide<+->\item Unfortunately, it's not in Text.Parser.
    \onslide<+->\item However, \IdrisFunction{>>=} is, so we'll be able to build
        it.
    \onslide<+->\item Next up: rewrite \IdrisFunction{chainl} to deal with
        levels.
\end{itemize}
\end{frame}

\frameAndSection{Questions}
\onslide<+->
\end{frame}

\end{document}

% vim: spell spelllang=en
